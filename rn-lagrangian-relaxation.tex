\documentclass{article}
\pagestyle{plain}
\setlength\textwidth{266.0pt}
\usepackage{amsmath}
\usepackage{amsthm}
\usepackage{amssymb}
\usepackage{mathrsfs}
\usepackage{graphicx}
\usepackage{epstopdf}
\usepackage{subfigure}
\usepackage{caption}
\usepackage{bm}
\usepackage{array}
\usepackage{multirow}
\usepackage[paperwidth=185mm,paperheight=260mm,text={148mm,220mm},left=21mm,top=25.5mm]{geometry}
\usepackage[round]{natbib}
\usepackage{booktabs}
\usepackage{algorithm,algpseudocode,amsmath}
\usepackage[colorlinks,linkcolor=blue,anchorcolor=blue,citecolor=blue]{hyperref}
\usepackage{color}
\usepackage{xcolor}
\usepackage{tikz}  
\usepackage{pdflscape}
 \newtheorem{thm}{Theorem}
\newtheorem{pro}{Proposition}
\newtheorem{lem}{Lemma}
\newtheorem{alg}{Algorithm}
\newtheorem{ass}{Assumption}
\newtheorem{coro}{Corollary}
\newtheorem{defin}{Definition}
\newtheorem{ob}{Observation}
\newtheorem{example}{Example}
\newtheorem{remark}{Remark}
%\newtheorem{proof}{Proof}
\usetikzlibrary{arrows,shapes,chains}  
\renewcommand{\algorithmicrequire}{\textbf{Input:}}  % Use Input in the format of Algorithm  
\renewcommand{\algorithmicensure}{\textbf{Output:}} % Use Output in the format of Algorithm  
\title {Review Notebook: Lagrangian Relaxation}
\author{CO2GENERATOR} 
\begin {document}
\maketitle 
\allowdisplaybreaks[4]

\section{Lagrangian Relaxation}
 
Consider the following linear integer programming problem(ILP),
\begin{align}
	(\textit{ILP}) \min \quad & \bm{c}^T\bm{x} \\
	s.t. \quad & \left\{\begin{aligned}
	& \bm{A}\bm{x} \leq \bm{b} \\
	& \bm{D}\bm{x} \leq \bm{d} \\
	& \bm{x} \in \mathbb{Z}^n
	\end{aligned} \right.
\end{align}
in which $\bm{D} \bm{x} \leq \bm{d}$ is $m$ constraints. Let $\bm{X}= \left\{\bm{x} \in \mathbb{R}^n \ | \ \bm{A}\bm{x} \leq \bm{b}\right\}$, $Conv(\bm{X}) = \left\{\bm{x} \in \mathbb{Z}^n \ | \ \bm{A}\bm{x} \leq \bm{b}\right\}$, and $\bm{u} = (u_1, ..., u_m)^T \in \mathbb{R}_+^m$, then the following problem
\begin{equation}
	 d(\bm{u}) = \min_{\bm{x} \in \bm{X}} \quad \mathcal{L}(\bm{x}, \bm{u}) := \bm{c}^T\bm{x} + \bm{u}^T(\bm{D}\bm{x} - \bm{d}) 
\end{equation}
 is defined as \textit{Lagrangian relaxation}, $\bm{u}$ is so called the \textit{Lagrangian multipliers}. And 
 \begin{equation}
 	(\textit{LD}) \max_{\bm{u} \in \mathbb{R}_+^m}d(\bm{u})
 \end{equation}
 is the \textit{Lagrangian Dual Problem}.
 
 \begin{thm} 
 	(Weak Duality) The following equation holds,
 	\begin{equation}
 		v(LD) \leq v(ILP)
 	\end{equation}
 \end{thm}

\begin{proof}
	For any $\bm{u} \in \mathbb{R}_+^m$ and $\bm{x} \in \bm{X}$ that $\bm{Dx} \leq \bm{d}$, we have
	\begin{equation}
		\bm{c}^T \bm{x} + \bm{u}^T (\bm{Dx} - \bm{d}) \leq \bm{c}^T\bm{x}
	\end{equation}
	Therefore,
	\begin{equation}
		v(LD) = \max_{\bm{u} \in \mathbb{R}_+^m} d(\bm{u}) \leq \min \left\{\bm{c}^T\bm{x} \ | \ \bm{D}\bm{x} \leq \bm{d}, \bm{x} \in \bm{X}\right\} = v(ILP)
	\end{equation}
\end{proof}

\begin{thm}
	(Strong Duality) Let $\bm{u}^* \in \mathbb{R}_+^m$, if the following conditions holds,
	\begin{enumerate}
		\item $\bm{x}^* \in \bm{X}$ is the optimal solution of problem \textit{LD}.
		\item $\bm{x}^*$ is the feasible solution of problem \textit{ILP}.
		\item the complementary slackness conditions holds, i.e. $\bm{u}^{*T}(\bm{D}\bm{x}^* - \bm{d}) = 0$.
	\end{enumerate} 
	then $\bm{x}^*$ is the optimal solution to \textit{ILP} and $v(ILP) = v(LD)$.
\end{thm}

\begin{proof}
	With condition 1 and 3, we have 
	\begin{equation}
		v(LD) \geq d(\bm{u}^*) = \bm{c}^T\bm{x}^* + \bm{u}^{*T}(\bm{D}\bm{x}^*-\bm{d}) = \bm{c}^T\bm{x}^*
	\end{equation}
	And with condition 2, 
	\begin{equation}
		\bm{c}^T\bm{x}^* \geq v(ILP)
	\end{equation}
	Then, we have that
	\begin{equation}
		v(LD) \geq \bm{c}^T\bm{x}^* \geq v(ILP)
	\end{equation}
	And we have already known the \textit{Weak Duality} holds,
	\begin{equation}
		v(ILP) \geq v(LD)
	\end{equation}
	Therefore,
	\begin{equation}
		v(ILP) = v(LD)
	\end{equation}
	and $\bm{x}^*$ is the optimal solution to problem \textit{ILP}.
\end{proof}


\section{Continuous Relaxation and Dual Relaxation}
 
 Consider the following general integer programming problem,
 \begin{align}
 	(\textit{IP}) \min \quad & f(\bm{x}) \\
	s.t. \quad & \left\{ 
	\begin{aligned}
	& g(\bm{x}) \leq \bm{b} \\
	& \bm{x} \in \bm{X} 
	\end{aligned} \right.
 \end{align} 
 where $\bm{X}$ is a finitely integers set and $\bm{b}$ is $m \times 1$ column vector. Then, the Lagrangian dual problem of \textit{IP} is,
 \begin{align}
 	(DIP) \max_{\bm{\lambda} \in \mathbb{R}_+^m} \min_{\bm{x} \in \bm{X}} f(\bm{x}) + \bm{\lambda}^T(g(\bm{x}) - \bm{b})
 \end{align}
 where $\bm{\lambda}$ is the Lagrangian multipliers. And the continuous relaxation of problem \textit{IP} is,
 \begin{align}
  	(\textit{CIP}) \min \quad & f(\bm{x}) \\
	 s.t. \quad & \left\{ 
	 \begin{aligned}
	 & g(\bm{x}) \leq \bm{b} \\
	 & \bm{x} \in Conv(\bm{X}) 
	 \end{aligned} \right.
 \end{align}
 where $Conv(\bm{X})$ is the convex hull of $\bm{X}$.

\begin{thm}
	The lower bound obtained by Lagrangian dual problem can always be better than continuous relaxation problem, i.e.
	\begin{equation} 
		v(DIP) \geq v(CIP)
	\end{equation}
\end{thm}

\begin{proof}
	Since $\bm{X} \subseteq Conv(\bm{X})$ and the strong duality theorem holds for convex optimization problems,
	\begin{align}
		v(DIP) & = \max_{\bm{\lambda} \in \mathbb{R}_+^m} \min_{\bm{x} \in \bm{X}} \mathcal{L}(\bm{x}, \bm{\lambda}) \\
		& \geq \max_{\bm{\lambda} \in \mathbb{R}_+^m} \min_{\bm{x} \in Conv(\bm{X})} \mathcal{L}(\bm{x}, \bm{\lambda}) \\
		& = v(CIP)
	\end{align}
	Therefore, $v(DIP) \geq v(CIP)$.
\end{proof}

In particular, for linear integer programming(ILP),
\begin{align}
	\textit{(ILP)} \min \quad & \bm{c}^T\bm{x} \\
	 s.t. \quad & \left\{ 
	\begin{aligned}
	& \bm{D}\bm{x} \leq \bm{d} \\
	& \bm{x} \in \bm{X}
	\end{aligned} \right.
\end{align}
one of the continuous relaxation can be,
\begin{align}
	\textit{(CILP)} \min \quad & \bm{c}^T\bm{x} \\
	s.t. \quad & \left\{ 
	\begin{aligned}
	& \bm{D}\bm{x} \leq \bm{d} \\
	& \bm{x} \in Conv(\bm{X})
	\end{aligned} \right.
\end{align}

\begin{thm}
	For linear integer programming, the lower bound obtained by Lagrangian dual problem is the same as continuous relaxation, i.e.
	\begin{equation}
		v(DILP) = v(CILP)
	\end{equation}
\end{thm}
\begin{proof}
	Assume $\bm{X} = \left\{\bm{x}^1, ..., \bm{x}^N\right\}$, then we have that
	\begin{align}
		v(DILP) &= \max_{\bm{\lambda} \geq 0} d(\lambda) \\
		&= \max_{\bm{\lambda} \geq 0} \min_{\bm{x} \in \bm{X}} \bm{c}^T\bm{x} + \bm{\lambda}^T(\bm{Dx} - \bm{d}) \\
		&= \max_{\bm{\lambda} \geq 0} \min_{i = 1 , ... , N} \bm{c}^T\bm{x}^i + \bm{\lambda}^T(\bm{D}\bm{x}^i - \bm{d}) \\
		& = \max \quad \eta \\
		& \quad \ \begin{aligned}
		s.t. \quad& \bm{c}^T\bm{x}^i + \bm{\lambda}^T(\bm{D}\bm{x}^i - \bm{d}) \geq \eta , i = 1, ..., N\\
		& \bm{\lambda} \geq 0, \eta \in \mathbb{R}
		\end{aligned} 
	\end{align}
	As we can see, problem (29)-(30) is a linear programming problem which has enormous number of constraints. And its dual problem is,
	\begin{align}
	\min \quad & \sum_{i=1}^{N}\mu_i \bm{c}^T\bm{x}^i \\
	s.t. \quad & \sum_{i=1}^{N}\mu_i(\bm{D}\bm{x}^i - \bm{d}) \leq 0 \\
	& \sum_{i=1}^{N}\mu_i = 1, \mu_i \geq 0
	\end{align}
	Let $\bm{x}^\prime = \sum_{i=1}^{N} \mu_i \bm{x}^i$, since $\sum_{i=1}^{N} \mu_i = 1, \mu_i \geq 0$, so $\bm{x}^\prime$ is the convex combination of $\bm{x}^i, i =1, ..., N$. And problem (32)-(34) is equivalent to the following problem,
	\begin{equation}
		\min \left\{\bm{c}^T\bm{x}^\prime \ | \ \bm{D}\bm{x} \leq \bm{d}, \bm{x}^\prime \in Conv(\bm{X})\right\}
	\end{equation}
	And it definitely is problem \textit{CILP}. Since the strong duality holds for linear programming, it can be easily find that $v(DILP) = v(CILP)$.
\end{proof}

\begin{coro}
	Assume $\bm{X} = \left\{\bm{x} \in \mathbb{Z}^n \ | \ \bm{Ax} \leq \bm{b} \right\}$ and $\bar{\bm{X}} = \left\{ \bm{x} \in \mathbb{R}^n \ | \ \bm{Ax} \leq \bm{b} \right\}$. It can be easily find that for problem \textit{ILP}, its linear programming relaxation,
	\begin{align}
		(\textit{LP})\min \left\{\bm{c}^T\bm{x} \ | \ \bm{Dx} \leq \bm{d}, \bm{x} \in \bar{\bm{X}}\right\}
	\end{align}
	the following inequality holds,
	\begin{equation}
		v(DILP) = v(CIP) \geq v(LP)
	\end{equation}
\end{coro}
 
\section{Dual Search} 

\subsection{Sub-gradient Search}
 
\subsection{Out-approximation Approach}

\subsection{Bundle Method}
 
% \newpage
% \bibliographystyle{chicago}
% \bibliography{reference}
\end {document} 