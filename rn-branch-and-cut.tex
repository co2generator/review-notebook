\documentclass{article}
\pagestyle{plain}
\setlength\textwidth{266.0pt}
\usepackage{amsmath}
\usepackage{amsthm}
\usepackage{amssymb}
\usepackage{mathrsfs}
\usepackage{graphicx}
\usepackage{epstopdf}
\usepackage{subfigure}
\usepackage{caption}
\usepackage{bm}
\usepackage{array}
\usepackage{multirow}
\usepackage[paperwidth=185mm,paperheight=260mm,text={148mm,220mm},left=21mm,top=25.5mm]{geometry}
\usepackage[round]{natbib}
\usepackage{booktabs}
\usepackage{algorithm,algpseudocode,amsmath}
\usepackage[colorlinks,linkcolor=blue,anchorcolor=blue,citecolor=blue]{hyperref}
\usepackage{color}
\usepackage{xcolor}
\usepackage{tikz}  
\usepackage{pdflscape}
 \newtheorem{thm}{Theorem}
\newtheorem{pro}{Proposition}
\newtheorem{lem}{Lemma}
\newtheorem{alg}{Algorithm}
\newtheorem{ass}{Assumption}
\newtheorem{coro}{Corollary}
\newtheorem{defin}{Definition}
\newtheorem{ob}{Observation}
\newtheorem{example}{Example}
\newtheorem{remark}{Remark}
%\newtheorem{proof}{Proof}
\usetikzlibrary{arrows,shapes,chains}  
\renewcommand{\algorithmicrequire}{\textbf{Input:}}  % Use Input in the format of Algorithm  
\renewcommand{\algorithmicensure}{\textbf{Output:}} % Use Output in the format of Algorithm  
\title {Review Notebook: Branch and Cut}
\author{CO2GENERATOR} 
\begin {document}
\maketitle 
\allowdisplaybreaks[4]

\section{Some Questions}

\noindent $\bullet$ How to use the Gomory's cutting plane method to solve the following integer programming problem?
\begin{align}
	\max \quad & 3x_1 - x_2 \\
	s.t. \quad & \left\{
	\begin{aligned}
		& 3x_1 - 2x_2 \leq 3 \\
		& 5x_1 + 4x_2 \geq 10 \\
		& 2x_1 + x_2 \leq 5 \\
		& x_1, x_2 \in \mathbb{Z}^+
	\end{aligned}\right.
\end{align}

\noindent $\bullet$ How to use the branch and bound method to solve the following integer programming problem?
\begin{align}
	\max \quad & 2x_1 + x_2 \\
	s.t. \quad & \left\{
	\begin{aligned}
		& x_1 + x_2 \leq 5 \\
		& -x_1 + x_2 \leq 0 \\
		& 6x_1 + 2x_2 \leq 21 \\
		& x_1, x_2 \in \mathbb{Z}^+
	\end{aligned}\right.
\end{align}

\section{Cutting Plane}

Consider following integer programming problem,
\begin{align}
\label{problem:ip}
\begin{aligned}
\min \quad & c^Tx \\
s.t. \quad & Ax = b \\
& x \in \mathbb{Z}_+^n
\end{aligned}
\end{align}
And its linear relaxation is,
\begin{align}
\label{linear-relaxation}
\begin{aligned}
\min \quad & c^Tx \\
s.t. \quad & Ax = b \\
& x \in \mathbb{R}_+^n
\end{aligned}
\end{align}

Suppose the optimal solution obtained by (\ref{linear-relaxation}) is $x^*$ and given an optimal LP basis $B$, problem (\ref{problem:ip}) can be rewrote as,
\begin{align}
\min \quad & c_B^T B^{-1} b + \sum_{j \in NB} \bar{c}_jx_j \\
s.t. \quad & (x_B)_i + \sum_{j \in NB} \bar{a}_{ij}x_j = \bar{b}_i, && i = 1, 2, ..., m \\
& x_j \in \mathbb{Z}_+^1, && j = 1, 2, ..., n 
\end{align}
where $NB$ is the set of nonbasic variables. If $\forall i = 1, 2, ..., m$,  $\bar{b}_i$ is integer, then we obtain the optimal solution. Otherwise, there must be at least one $\bar{b}_i$ is fractional,
\begin{equation}
(x_B)_k + \sum_{j \in NB} \bar{a}_{kj}x_j = \bar{b}_k \label{fractional}
\end{equation}

 Since $x_j \geq 0$,
\begin{equation}
(x_B)_k + \sum_{j \in NB} \lfloor \bar{a}_{kj} \rfloor x_j \leq (x_B)_k + \sum_{j \in NB} \bar{a}_{kj}x_j  = \bar{b}_k
\end{equation}

Since $x_j$ should be integer,
\begin{equation}
(x_B)_k + \sum_{j \in NB} \lfloor \bar{a}_{kj} \rfloor x_j \leq \lfloor \bar{b}_k  \rfloor \label{vi}
\end{equation}

 Obviously, (\ref{vi}) is a valid inequality. Then, use (\ref{fractional}) and (\ref{vi}), we have
\begin{equation}
\sum_{j \in NB} (\bar{a}_{kj} - \lfloor \bar{a}_{kj} \rfloor) x_j \geq \bar{b}_k -  \lfloor \bar{b}_k  \rfloor \label{sub}
\end{equation}

Let $f_{kj} = \bar{a}_{kj} - \lfloor \bar{a}_{kj} \rfloor$, $g_k =  \bar{b}_k -  \lfloor \bar{b}_k\rfloor $, then we obtain the famous \textit{Chv\'atal-Gomory Cut},
\begin{equation}
-\sum_{j \in NB} f_{kj} x_j \leq -g_k \label{cg-cut}
\end{equation}

 It is interesting. For the optimal solution $x^*$ of linear relaxation, the non-basic variables $x_j$ is zero and can't satisfy inequality (\ref{cg-cut}). Thus, such fractional solution will be cut off.
 
 \begin{remark}
 	
 \end{remark}


\newpage
\section{Branch and Bound}

Take the following problem as an example,
\begin{align}
\min \quad & 4x_1 + 9x_2 + 6x_3 \\
s.t. \quad & 5x_1 + 8x_2 + 6x_3 \geq 12 \\
& x_1, x_2, x_3 \in \left\{0, 1\right\}
\end{align}
The optimal solution is $(1, 1, 0)$, objective is $13$. And its linear relaxation,
\begin{align}
\min \quad  4x_1 + 9x_2 + 6x_3 \qquad\qquad  \\
s.t. \quad \left\{
\begin{aligned}
& \quad 5x_1 + 8x_2 + 6x_3 \geq 12 \\
& \quad 0 \leq x_1 \leq 1 \\
& \quad 0 \leq x_2 \leq 1 \\
& \quad 0 \leq x_3 \leq 1 \\
& \quad x_1, x_2, x_3 \in \mathbb{R}
\end{aligned} \right.
\end{align}
The optimal solution is $(1, 0.125, 1)$, objective is $11.125$. Thus, \textit{Branch and Bound} can be illustrated as,

\begin{figure}[htbp]
	\centering
	\includegraphics[width=0.8\linewidth]{./figure/rn-b-a-c-tree.png}
	\caption{Search Tree of Above Problem}
	\label{fig:bb}
\end{figure}

\noindent The generic branch and bound algorithm can be described as,
\begin{algorithm}[htbp]
	\caption{Generic Branch and Bound Algorithm}
	\label{alg:bb}
	\begin{algorithmic}[1]
		\State Set $U = +\infty$ and break the original problem into two subproblems
		\State Select an active subproblem
		\State If the subproblem is infeasible, delete it; otherwise, compute $b(F)$ for the corresponding subproblem as a lower bound.
		\State If $b(F) \geq U$, delete the subproblem
		\State If $b(F) < U$, either obtain an optimal solution to subproblem, or break the corresponding subproblem into further subproblems, which are added to the list of active subproblems
		\State Repeat 2$\sim$5, until there is no active subproblem.
	\end{algorithmic}  
\end{algorithm} 

\section{Branch and Cut}


\section{Conclusion}


% \newpage
% \bibliographystyle{chicago}
% \bibliography{reference}
\end {document} 