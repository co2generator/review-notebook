\documentclass{article}
\pagestyle{plain}
\setlength\textwidth{266.0pt}
\usepackage{amsmath}
\usepackage{amsthm}
\usepackage{amssymb}
\usepackage{mathrsfs}
\usepackage{graphicx}
\usepackage{epstopdf}
\usepackage{subfigure}
\usepackage{caption}
\usepackage{bm}
\usepackage{array}
\usepackage{multirow}
\usepackage[paperwidth=185mm,paperheight=260mm,text={148mm,220mm},left=21mm,top=25.5mm]{geometry}
\usepackage[round]{natbib}
\usepackage{booktabs}
\usepackage{algorithm,algpseudocode,amsmath}
\usepackage[colorlinks,linkcolor=blue,anchorcolor=blue,citecolor=blue]{hyperref}
\usepackage{color}
\usepackage{xcolor}
\usepackage{tikz}  
\usepackage{pdflscape}
 \newtheorem{thm}{Theorem}
\newtheorem{pro}{Proposition}
\newtheorem{lem}{Lemma}
\newtheorem{alg}{Algorithm}
\newtheorem{ass}{Assumption}
\newtheorem{corr}{Corollary}
\newtheorem{defin}{Definition}
\newtheorem{ob}{Observation}
\newtheorem{example}{Example}
\newtheorem{remark}{Remark}
%\newtheorem{proof}{Proof}
\usetikzlibrary{arrows,shapes,chains}  
\renewcommand{\algorithmicrequire}{\textbf{Input:}}  % Use Input in the format of Algorithm  
\renewcommand{\algorithmicensure}{\textbf{Output:}} % Use Output in the format of Algorithm  
\title {Review Notebook: Linear Programming}
\author{CO2GENERATOR} 
\begin {document}
\maketitle 
\allowdisplaybreaks[4]

 

\section{Some Questions}

In a maximization linear programming problem, 

\begin{itemize}
	\item why the rule that all $\sigma_j = c_j - z_j \leq 0$, can be used to determine whether the current basic feasible solution is optimal?
	\item suppose that we already have the primal optimal solution, how to obtain the dual solution directly? 
	\item if we newly add a constraint, how to operate a warm start rather than re-optimize? 
\end{itemize}

These questions have puzzled me for a long time.

\newpage
\section{Simplex Method}

The simplex algorithm proceeds as follows:

\begin{itemize}
	\item \textbf{Step 1} Convert the LP to the standard form.
	\item \textbf{Step 2} Obtain basic feasible solution(bfs) if possible from the standard form.
	\item \textbf{Step 3} Determine whether the current bfs is optimal.
	\item \textbf{Step 4} If the current bfs is not optimal, determine which nonbasic variable should become a basic variable and which basic variable should become a nonbasic variable to find a new bfs with a better objective function value.
	\item \textbf{Step 5} Use EROS(elementary row operations) to find the new bfs with the better objective function value. Go back to \textbf{Step 3}.
\end{itemize}

\begin{example}
	The following shows a simple case:
	\begin{align}
		&\max \quad  2x_1 + x_2 \\
		& s.t. \left\{ \begin{aligned}
		 \ & 5x_2 \leq 15 \\
		& 6x_1 + 2x_2 \leq 24 \\
		& x_1 + x_2 \leq 5 \\
		& x_1, x_2 \geq 0
		\end{aligned} \right.
	\end{align}
	The simplex tableau has been shown as below:
	\begin{table}[htbp]
		\centering
		\label{tab:simpelx}
		\begin{tabular}{@{}ccccccccc@{}}
			\toprule
			\multicolumn{3}{c}{$c_j$}      & 2       & 1    & 0    & 0    & 0     & \multirow{2}{*}{$\theta$} \\ \cmidrule(r){1-8}
			$c_B$     & $x_B$     & $b$       & $x_1$    & $x_2$ & $x_3$ & $x_4$ & $x_5$  &                        \\ \midrule
			0        & $x_3$     & 15      & 0       & 5    & 1    & 0    & 0     & -                      \\
			0        & $x_4$     & 24      & [6] & 2    & 0    & 1    & 0     & 4                      \\
			0        & $x_5$     & 5       & 1       & 1    & 0    & 0    & 1     & 5                      \\ \midrule
			\multicolumn{3}{c}{$c_j-z_j$} & 2       & 1    & 0    & 0    & 0     &                        \\ \midrule
			0        & $x_3$     & 15      & 0       & 5    & 1    & 0    & 0     & -                      \\
			2        & $x_1$     & 4       & 1       & 1/3  & 0    & 1/6  & 0     & 4                      \\
			0        & $x_5$     & 1       & 0       & [2/3]  & 0    & -1/6 & 1     & 3/2                    \\ \midrule
			\multicolumn{3}{c}{$c_j-z_j$} & 0       & 1/3  & 0    & -1/3 & 0     &                        \\ \midrule
			0        & $x_3$     & 15/2    & 0       & 0    & 1    & 5/4  & -15/2 &                        \\
			2        & $x_1$     & 7/2     & 1       & 0    & 0    & 1/4  & -1/2  &                        \\
			1        & $x_2$     & 3/2     & 0       & 1    & 0    & -1/4 & 3/2   &                        \\ \midrule
			\multicolumn{3}{c}{$c_j-z_j$} & 0       & 0    & 0    & -1/4 & -1/2  &                        \\ \bottomrule
		\end{tabular}%
	\end{table}
\end{example}

When a starting bfs is not readily apparent, the Big-M method or the Two-phase simplex method may be used to solve the problem.
\begin{itemize}
	\item \textbf{Big-M Method}: we can introduce some artificial variables to construct starting bfs. Since in the final solution, the value of these arbitrary variables must be $0$, a very large number $M$ is introduced as the coefficient of these variables in the objective function. 
	\item \textbf{Two-phase Simplex Method}: because of the difficulty on numerical computation, Big-M method is not suitable for programming. We add artificial variables to the same constraints as we did in the Big-M method. And construct the Phase I LP, in which the objective is to minimize the sum of all artificial variables. At the completion of Phase I, if the optimal objective is not $0$, it confirmedly demonstrates that the original LP is infeasible.(Maybe in CPLEX/Gurobi, the same process is employed to detect the feasibility.) Otherwise, the optimal solution is $0$, then introduce the original LP's objective function to continue the simplex process.   
\end{itemize}

\subsection{The Optimality Condition}

In the process of simplex algorithm, we use the $\sigma_j = c_j - z_j$ to find entering variable. For a maximization problem, if $\forall j, \sigma_j <0$, we obtain the optimal solution. Why it is true? 

\begin{proof}
	For a standard maximization LP,
	\begin{align}
		& \max \quad \sum_{j=1}^{n} c_jx_j \\
		& s.t. \quad \left\{
		\begin{aligned}
			& \sum_{j=1}^{n} P_j x_j = b \\
			& x_j \geq 0, (j = 1, ..., n)
		\end{aligned} \right.
	\end{align}
	Suppose we have got a basic feasible solution,
	\begin{equation}
		X^{(0)} = (x_1^{(0)}, ..., x_m^{(0)}, 0, ..., 0)^T
	\end{equation}
	Then, we have
	\begin{equation}
		\sum_{i=1}^m P_i x_i^{(0)} = b \label{eq1}
	\end{equation}
	And $P_1, ..., P_m$ is a basis. So, for the other vector $P_j$, they can be represented as the following linear convex combination:
	\begin{equation}
		P_j  = \sum_{i=1}^{m} a_{ij}P_i
	\end{equation}  
	We can introduce a positive number $\theta$ (in fact, this is the step length), then
	\begin{equation}
		\theta(P_j - \sum_{i=1}^{m}a_{ij}P_i) = 0 \label{eq2}
	\end{equation}
	And (\ref{eq1}) + (\ref{eq2}), we have
	\begin{align}
		\sum_{i=1}^{m} P_i x_i^{(0)} + \theta(P_j - \sum_{i=1}^{m} a_{ij}P_i) = b \\
		\sum_{i=1}^{m}(x_i^{(0)} - \theta a_{ij}) P_i + \theta P_j = b
	\end{align}
	Thus, for the constraints equation system,
	\begin{equation}
		\sum_{j=1}^{n} P_j x_j = b
	\end{equation} 
	We have got a feasible solution,
	\begin{equation}
		X^{(1)} = (x_1^{(0)} - \theta a_{1j}, ..., x_m^{(0)} - \theta a_{mj}, 0, ..., \theta, ..., 0)
	\end{equation}	
	And clearly, it is a basic feasible solution. Thus, 
	\begin{equation}
		z^{(0)} = \sum_{i=1}^{m} c_i x_i^{(0)}
	\end{equation}
	\begin{align}
		z^{(1)} & = \sum_{i=1}^{m} c_i (x_i^{(0)} - \theta a_{ij}) + \theta c_j \\
		& = \sum_{i=1}^{m} c_ix_i^{(0)} + \theta (c_j - \sum_{i=1}^{m}c_i a_{ij})
	\end{align}
	Let,
	\begin{equation}
		\sigma_j = c_j - \sum_{i=1}^{m} c_i a_{ij}
	\end{equation}
	in which $c_j$ is the \textit{cost} in the objective function. So, $\sigma_j$ is often referred as \textit{reduced cost}. And if $\forall j, \sigma_j \leq 0$, then $z^{(1)} \leq z^{(0)}$, $z^{(0)}$ is the optimal solution. And if there exists nonbasic variables $x_j$ of which the reduced cost $\sigma_j = 0$, then this LP has infinite number of optimal solution. And if there exists $\sigma_j > 0$ and $P_j \leq 0$, then $x_i^{(0)} - \theta a_{ij}$ can be infinitely large, this LP is unbounded.
\end{proof}


\subsection{Matrix Description}

The following shows a maximization LP problem,
\begin{align}
& \max \quad z = \bm{c}^T\bm{x} \label{LP-obj} \\
& s.t. \quad \left\{ 
\begin{aligned}
& \bm{A}\bm{x} \leq \bm{b} \\
& \bm{x} \geq 0
\end{aligned} \right. \label{LP-cons}
\end{align}
It can be convert to the standard form as,
\begin{align}
	& \max \quad z = \bm{c}_B^T\bm{x}_B + \bm{c}_N^T \bm{x}_N + 0 \bm{x}_s \\
	& s.t. \quad \left\{ 
	\begin{aligned}
		& \bm{B}\bm{x}_B + \bm{N} \bm{x}_N + \bm{I}\bm{x}_s = \bm{b} \\
		& \bm{x}_B, \bm{x}_N, \bm{x}_s \geq 0
	\end{aligned} \right.
\end{align}
where $\bm{x}_B$ are the basic variables, $\bm{x}_N$ are the nonbasic variables and $\bm{x}_s$ are the slack variables. And the iterated simplex tableau is,

\begin{table}[htbp]
	\centering
	\begin{tabular}{@{}cccccc@{}}
		\toprule
		\multicolumn{3}{c}{Content} & $\bm{x}_B$ & $\bm{x}_N$      & $\bm{x}_s$  \\ \midrule
		$\bm{c}_B$      & $\bm{x}_B$      & $\bm{B}^{-1}\bm{b}$      & $I$  & $\bm{B}^{-1}\bm{N}$      & $\bm{B}^{-1}$   \\ \midrule
		\multicolumn{3}{c}{$c_j-z_j$}     & 0  & $\bm{c}^T_N-\bm{c}^T_B\bm{B}^{-1}N$ & $-\bm{c}^T_BB^{-1}$ \\ \bottomrule
	\end{tabular}%
\end{table}

If the $\bm{B}$ is the optimal basis, 
\begin{equation}
	\bm{c}^T_N - \bm{c}^T_B\bm{B}^{-1}\bm{N} \leq 0
\end{equation} 
\begin{equation}
	-\bm{c}^T_B\bm{B}^{-1} \leq 0
\end{equation}

And we have, 
\begin{equation}
	\bm{c}^T_B - \bm{c}^T_B\bm{I} = 0 
\end{equation}

Thus, 
\begin{equation}
	\bm{c}^T - \bm{c}^T_B\bm{B}^{-1}\bm{A} \leq 0
\end{equation}
\begin{equation}
	-\bm{c}^T_B \bm{B}^{-1} \leq 0
\end{equation}

Let $\bm{Y}^T = \bm{c}^T_B\bm{B}^{-1}$, then
\begin{align}
 	\begin{aligned}
 		\bm{c}^T - \bm{Y}^T\bm{A} \leq 0 \\
 		-\bm{Y}^T \leq 0
 	\end{aligned}
\end{align}

Let us take look at the dual problem of (\ref{LP-obj})-(\ref{LP-cons}) is,
\begin{align}
	& \min \quad w = \bm{b}^T\bm{y} \\
	& s.t. \quad \left\{
	\begin{aligned}
		& \bm{A}^T\bm{y} \geq \bm{c} \\
		& \bm{y} \geq 0
	\end{aligned} \right.
\end{align}

It can be find that $\bm{Y}$ is the feasible solution of such dual problem. And,
\begin{equation}
	w = \bm{b}^T\bm{y} = \bm{c}_B^T\bm{B}^{-1}\bm{b} = z
\end{equation}

So, is $\bm{Y}$ the optimal solution to dual problem?

\begin{proof}
	Let $\hat{\bm{x}} = (x_1, ..., x_n)^T$ be a feasible solution to primal problem, $\hat{\bm{y}} = (y_1, ..., y_m)^T$ be a feasible solution to dual problem. And,
	\begin{equation}
		\bm{c}^T \hat{\bm{x}}  = \bm{b}^T \hat{\bm{y}}
	\end{equation}
	So, assume $\bm{x}^* = (x_1, ..., x_n)^T$ is the optimal solution to primal problem, $\bm{y}^* = (x_1, ..., x_m)^T$ is the optimal solution to dual problem. Then, we have:
	\begin{equation}
		\bm{c}^T\hat{\bm{x}} \leq \bm{c}^T \bm{x}^*, \bm{b}^T \bm{y}^* \leq \bm{b}^T \hat{\bm{y}}
	\end{equation}
	And, 
	\begin{equation}
		\bm{c}^T\hat{\bm{x}} = \bm{b}^T\hat{\bm{y}}, \bm{c}^T\bm{x}^* \leq \bm{b}^T \bm{y}^*
	\end{equation}
	Thus,
	\begin{equation}
			\bm{c}^T\hat{\bm{x}} = \bm{c}^T \bm{x}^* = \bm{b}^T \bm{y}^* = \bm{b}^T \hat{\bm{y}}
	\end{equation}
	Therefore, $\bm{Y}$ is the optimal solution of dual problem.
\end{proof}

The dual problem solution can be obtained directly through the optimal tableau. It's awesome!


 \section{Dual Simplex Method}

When we use the simplex method to solve a max problem, we begin with a primal feasible solution(but not dual feasible). Through a sequence of simplex pivots, we maintain primal feasibility and obtain an optimal solution when dual feasibility is attained. In many situations, however, we can maintain the dual feasibility, and it is so called the \textit{dual simplex method}. 

% \newpage
% \bibliographystyle{chicago}
% \bibliography{reference}
\end {document} 